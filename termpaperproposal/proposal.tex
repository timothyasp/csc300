% Term paper proposal template - Ilona Sparks
% CSC 300: Professional Responsibilities
% Dr. Clark Turner

% One Column Format
\documentclass[11pt]{article}

\usepackage{setspace}
\usepackage{url}

%%% PAGE DIMENSIONS
\usepackage{geometry} % to change the page dimensions
\geometry{letterpaper}


\begin{document}

\title{\vfill "Credit Rating Agencies - The Companies Behind the Smoking Gun" Proposal} %\vfill gives us the black space at the top of the page
\author{
Timothy Asp \vspace{10pt} \\
CSC 300: Professional Responsibilities  \vspace{10pt} \\
Dr. Clark Turner \vspace{10pt} \\
}
\date{04/15/11} %Or use \Today for today's Date

\maketitle

\vfill  %in combination with \newpage this forces the abstract to the bottom of the page
\begin{abstract}
In the years leading up to the market crash of 2008, the United States housing market was booming and housing prices were on the rise.  This tempted many individuals to live outside their means and take on mortgages that they could not feasibly afford.  Banks enabled this risky behaviour by lending money without any proof they could pay it back.  This "bubble," as many called it, all came crashing down in September of 2008. This burst in the housing market brought the United States into one of the worst financial crisis since the Great Depression. Companies such as Fannie Mae and Freddie Mac, AIG, General Motors, and others were on the brink of failure and many individuals lost their homes to foreclosure or bankruptcy. Overall, the market dropped 41 percent over an eight week period. \cite{marketWatch}

Recently, the cause of this market crash was determined after a two-year federal investigation of the executives and important individuals in the financial sector. They found that the tipping point was when rating agencies such as Standard \& Poor's and Moody's received pressure from eight major banks to gave these bad mortgages a good rating.  \cite{ratingEthics} This brings up a clear ethical dilemma: should private companies, like S\&P and Moody's, be in charge of the rating and regulation of these financial markets when a clear conflict of interest exists.  
\end{abstract}

\thispagestyle{empty} %remove page number from title page, but still keep it as pg #1
\newpage

%%%%%%%%%%%%%%%%%%%%
%%% Known Facts  %%%
%%%%%%%%%%%%%%%%%%%%
\section{Facts}
\begin{enumerate}
\item Rating agencies such as Standard \& Poor's, Fitch, and Moody's rate "credit worthiness" based on a rating scale, such as 'AAA', 'BBB', etc. \cite{CivilLiability}
\item The SEC (Securities and Exchange Commission) regulates these rating agencies under what it calls "nationally recognized statistical rating organizations" (NRSRO's) \cite{CivilLiability}
\item The three agencies, Standard \& Poor's, Moody's Investor Service and Fitch Ratings control about 40 percent of the market. \cite{smoothing}
\item These agencies are paid by the issuers they rate, meaning the banks, investors and mortgage brokers. \cite{gatekeepers}
\item During the five years leading up to the housing market crash, mortgages were packaged as AAA rated securities and safe as government bonds. Of these packaged securities, 90 percent of them were downgraded to junk status between 2006 and 2007. \cite{ratingEthics}
\item Eight banks, including Goldman Sach's, JPMorgan, and UBS AG pressured these rating agencies to "weaken their standards" in order to boost business and achiever greater market share \cite{ratingEthics}
\item Over the first half of the 2000's, these rating agencies, specifically Moody's which is the only publicly traded agency, saw its revenues increase by over 500 percent.  \cite{gatekeepers}
\end{enumerate}

%%%%%%%%%%%%%%%%%%%%%%%%%
%%% Research Question %%%
%%%%%%%%%%%%%%%%%%%%%%%%%
\section{Research Question}
Is it ethical for companies to rate and act as "gatekeepers" for industries in which they stand to benefit/profit based on the ratings they give?

%%%%%%%%%%%%%%%%%%%%%%%%%%%%%%%%%%%%%%%%%%%%%%
%%% Extant Arguments from External Sources %%%
%%%%%%%%%%%%%%%%%%%%%%%%%%%%%%%%%%%%%%%%%%%%%%
\section{Extant arguments}
It is not ethical to have private rating agencies rate and package credit:
\begin{itemize}
\item These rating agencies reports are not accurate due to outside pressure from banks and poor rating standards.\cite{ratingEthics} \cite{hRatingEthics} \cite{govtReport}
\item Because the rating agencies are paid for their ratings by the banks who give them the credit to rate, there is a conflict of interest. \cite{gatekeepers}
\item These rating companies performance before and after the crash brings up questions of their success and validity in rating this credit. \cite{ratingEthics} \cite{wpMoodies}
\end{itemize}
It is ethical to have private rating agencies rate and package credit:
\begin{itemize}
\item The SEC regulates and has given approval to the three major rating agencies under NRSRO. \cite{CivilLiability}
\end{itemize}

%%%%%%%%%%%%%%%%%%%%%%%%%%%
%%% Analytic principles %%%
%%%%%%%%%%%%%%%%%%%%%%%%%%%
\section{Applicable analytic principles}
\begin{itemize}
\item Using the code of ethics from Goldman Sachs, there must be "Fair and Ethical Competition - ... No one at the firm may seek competitive advantage through illegal or unethical business practices" \cite{goldmanEthics}
\item They "shall act consistently with the public interest" \cite{SEcode}
\item When doing any kind of business, one should treat people as an end, not a means to an end \cite{kant}
\item "Disclose to all concerned parties those conflicts of interest that cannot reasonably be avoided or escaped" \cite{SEcode}
\item Do "not engage in deceptive financial practices such as bribery, double billing, or other improper financial practices. 
\end{itemize}
%%%%%%%%%%%%%%%%%%%%%%%%%%%%%%%%%%%%%%%
%%% Abstract your Expected Analysis %%%
%%%%%%%%%%%%%%%%%%%%%%%%%%%%%%%%%%%%%%%
\section{Abstract of Expected Analysis}
% Give a short abstract of the basics you expect to analyze and present in your paper. Divide it into sections that make sense for your work.

% One way would be to: a) start with deontological perspectives as a section where you analyze those arguments based on the inherent ethics of the act itself rather than the results or trade-offs; then, b) use a utilitarian perspective and list the appropriate analyses of the trade-offs and stakeholders to define the most desired results and how to get them. Be explicit about the trade-offs (what value is balanced against what other value, which stakeholders win, which stakeholders lose...) What is the "utility" in "utilitarian" in your case - what value do you want to advance the most (derived from the general utilitarian "happiness")? How do you maximize (or optimize) it?

% Note that the SE Code should be the center of your ethical analysis (and remember that it includes both deontological and utilitarian [and more] principles you can utilize). Estimate where you'll end up for your answer (you can change your mind in the final paper!). Keep referencing sources for any additional facts, quotes, or other information you might use here. \cite{handout}

\begin{itemize}
\item The US Financial Industry: An Overview
\begin{itemize}
   \item What are Mortgage Backed Securities and what are all the terms thrown around in the news?
   \item How is debt rated, traded, and profited from?
   \item How did this all end up almost destroying the US economy and costing the US government around 2 trillion?
\end{itemize}
\item Who are the key players in the financial world?
\begin{itemize}
   \item The People - Who take out the loans
   \item The Banks - Lend the money
   \item The Rating Agencies - Categorize and package the debt
   \item The Others - What happens after and who takes it over.
\end{itemize}
\item The Problem with the current system
\begin{itemize}
   \item Talk about the 2008 financial collapse and the housing bubble that burst.
   \item Show how the people involved were the cause of it.
\end{itemize}
\item Because the actions taken by individuals in the financial industry directly affect the everyday lives of most people, those involved must be held to a higher ethical standard.
\begin{itemize}
   \item Rating agencies must be reevaluated and heavily regulated
   \item Banks must treat people as ends, not as a means to an end \cite{kant}
   \item Because this is an issue that heavily affects a wide range of people, most of which do not understand the difference between the stock market and the supermarket, normal standards of ethics are not enough to make sure past actions are not repeated.
\end{itemize}
\end{itemize}
\newpage
% cite all the references you want in your annotated bibliography that you cite in the paper
\nocite{texTemp}
\nocite{BibTex}
\nocite{BibMang}
\nocite{bibStyle}

\bibliographystyle{IEEEannot}

\bibliography{proposal}
\end{document}

% template taken from github.com/mguis/csc300.git 
% used for Dr. Clark Turner's CSC 300 - professional responsibilities class
% all written content is researched and written by Timothy Asp unless other wise noted

% Two Column Format
\documentclass[11pt]{article}


% import some fancy packages
\usepackage{float}
\usepackage{graphicx}
\usepackage{setspace}
\usepackage{framed}
\usepackage{ulem}
\usepackage{url}
\usepackage{multicol}

%%% PAGE DIMENSIONS
\usepackage{geometry} % to change the page dimensions
\geometry{letterpaper}

\begin{document}

\title{\vfill Behind the Smoking Gun: Credit Rating Agencies and the Financial Crisis} %\vfill gives us the black space at the top of the page
% Awww yeea that's me
\author{
Timothy Asp\vspace{10pt} \\
CSC 300: Professional Responsibilities\vspace{10pt} \\
Dr. Clark Turner\vspace{10pt} \\
}

% paper release date
\date{May 27, 2011}

\maketitle

%in combinaion with \newpage this forces the abstract to the bottom of the page
\vfill
% start up that abstract
\begin{abstract}
In January 2011, the Financial Crisis Inquiry Commission (FCIC) released a report outlining their research and investigation into the recent financial crisis.  The commission concluded that the crisis was avoidable, citing 10 main causes: of which they describe the credit rating agencies as being ``essential cogs in the wheel of financial destruction.''\cite[p.~417-418]{govtReport}\cite[p.~xxv]{govtReport}  The rating agencies described are Standard \& Poor's, Moody's Investors Service, and Fitch Ratings.  The report shows that these agencies, specifically Standard \& Poor's and Moody's, received pressure from financial firms to give skewed ratings \cite{ratingEthics, hRatingEthics} and relied on outdated and flawed computer models in their rating of investments, specifically mortgage-backed securities (MBS) and collatorized debt obligations (CDO). \cite[p.~xxv]{govtReport}  This brings up an ethical dilemma: is it ethical for Standard \& Poor's and Moody's to rate and act as gatekeepers for industries in which they stand to benefit based on the ratings they give?  Because the financial collapse sent the United States into the largest recession since the Great Depression, and the rating agencies were determined to be ``key enablers'' to this financial crisis, they should not be solely in-charge of rating debt and securities. 
\end{abstract}

\thispagestyle{empty} %remove page number from title page
\newpage


%Create a table of contents with all headings of level 3 and above.
%http://en.wikibooks.org/wiki/LaTeX/Document_Structure#Table_of_contents has
%info on customizing the table of contents
\thispagestyle{empty}  %Remove page number from TOC
\tableofcontents

\newpage

%end the 1 column format

% this allows us to specify sections to be single or multi column so that things
% like title page and table of contents are single column
%start 2 column format
\begin{multicols}{2}
%Start numbering first page of content as page 1
\setcounter{page}{1}
%%%%%%%%%%%%%%%%%%%%
%%% Known Facts  %%%
%%%%%%%%%%%%%%%%%%%%
\section{Facts}

In order to understand and appreciate the impact Standard \& Poor's and Moody's had on the financial crisis, it is necessary to have a basic knowledge of the 2007 - 2009 financial crisis and the role the credit rating agencies had in it.  To aid with this understanding, the following sections lay out the facts regarding the financial crisis and credit rating agencies.


\subsection{Financial Crisis}
In Fall 2008, the United States experienced the worst financial crisis since the Great Depression in 1930.  The FCIC report states that over 26 million people were unemployed, four million families lost homes due to foreclosure and nearly \$11 trillion in household wealth was lost as a result of the financial collapse.\cite[p.~xv]{govtReport}  Overall, the market dropped 41\% over an eight week period.\cite{marketWatch}  It was the collapse of the housing bubble in late 2007, that sent the US into this financial crisis.\cite[p.~xxv]{govtReport} 


From the late 1990's to the early 2000's, the US housing market was booming and housing prices rose 124\% between 1997 and 2006. \cite{economistCreditCrunch}  This tempted many individuals to live outside their means and take on mortgages that they could not feasibly afford.  According to the FCIC report, nearly one in 10 mortgage borrowers in 2005 and 2006 took out option ARM (Adjustable Rate Mortgage) loans, which meant they could choose to make payments so low that their mortgage balances rose every month. \cite[p.~xx]{govtReport} It wasn't just individuals engaging in this risky financial behavior, banks and mortgage lenders acted as enablers by lending interest only loans, often requiring little or no proof of ability to repay. \cite[p.~xxiii]{govtReport}  In 2006, one-fifth of new mortgages were subprime, meaning they were given to people with a low credit rating or those who may have difficultly paying it back. \cite{economistCreditCrunch}  These subprime mortgages then began to find their way into mortgage-backed securities which ultimately became the tipping point for the collapse of the housing bubble. \cite[p.xxv]{govtReport}
\newline

\textbf{Mortgage-backed Securities}
\newline

Mortgage-backed securities are defined by the Securities and Exchange commission as ``debt obligations that represent claims to the cash flows from pools of mortgage loans, most commonly on residential property.'' \cite{secMBS} These loans undergo a process called securitization, which means they are pooled together, packaged, and rated.  This is done to minimize risks for investors by spreading the investment across multiple mortgages rather than relying on a single borrower as your investment's security. \cite{MBSInfo}

After the process of securitization, the investment banks who created these securities sell shares, called ``tranches,'' to investors who then can collect varying dividends from the interest payments on the mortgage.  From this comes numerous derivatives such as CDO's, CDS's, and a slew of other acronyms that investment banks and investors used both to increase earnings on securities and decrease risk. \cite{MBSInfo}\cite[p.73]{govtReport} To put it in the words of policy makers in the 1990's when this securitization process was first described to them, ``This stuff is so complicated how is anybody going to know? How are the buyers going to buy?'' \cite[p.~68]{govtReport}  

Their answer to this question was Credit Rating Agencies.

\subsection{Credit Rating Agencies}
In the US today, there are three primary credit rating agencies, Standard \& Poor's, Moody's Investor Service and Fitch Ratings.  These three control almost all of the market share; with Standard and Poor's and Moody's controlling 40\% each and Fitch controlling around 14\%. \cite{wpMoodies}  The Securities and Exchange Commission (SEC) regulates these rating agencies under what it calls ``nationally recognized statistical rating organizations'' (NRSRO's) \cite{CivilLiability} 

As mentioned in the previous section, when a bank wants to issue a MBS, they have to have it rated on its ``credit worthiness,'' which is simply the likelihood that the borrower will repay.  They rate this ``credit worthiness'' based on a rating scale: `AAA' being the highest (almost no chance of default) and `D'(in default.)  \cite{CivilLiability}  This scale was created to make it easier for investors to understand the quality of their investments and save time researching and checking on each and every part of a security.  From 2000 to 2007, Moody's rated nearly 45,000 mortgage-related securities as triple-A. \cite[p.~xxv]{govtReport} In 2005 and 2006, thousands of securities packaged mostly with subprime mortgages were given the AAA rating, deeming them safe as government bonds.  Of these packaged securities, 83\% were downgraded to junk status between 2006 and 2007. \cite[p.~xxv]{govtReport}\cite{ratingEthics}  

Unlike most regulatory or semi-regulatory agencies, all three credit rating agencies are for-profit businesses.  Standard \& Poor's and Fitch's Ratings are private companies, and Moody's is publicly traded.  Their business model changed from a subscription based system to an ``issuer pays'' model in the 1970's. \cite{CivilLiability, gatekeepers}  This means that they are charging the issuers of the bonds - meaning the banks, investors and mortgage brokers.  From 2000 to 2007, Moody's investment services saw their revenues grow from \$602 million to over \$2 billion. Their profits margins grew also during those years, from 26\% to 37\% \cite[p.149]{govtReport}.  In four years, 2002 - 2006, both Standard \& Poor's and Moody's gross revenue quadrupled.  According to the FCIC, Eight banks, including Goldman Sach's, JPMorgan, and UBS AG pressured these rating agencies to ``weaken their standards'' in order to boost business and achieve greater market share. \cite{ratingEthics} This was done primarily through market pressure made by the banks threatening to go to competitors unless they gave them a rating that the banks expected, which was usually higher than it deserved. \cite[p.210]{govtReport}

\subsection{``Quants'' and Their Models}

Coupled with this weakened standard of ratings, the FCIC concluded that outdated and flawed computer models contributed to the crisis. \cite[p.~xxv]{govtReport} The New York Times, as well as the FCIC, also noted that Quantitative Analysts for the major banks and rating agencies were creating overly optimistic models by building them from skewed or incorrect data. \cite{nyTimesQuants}\cite[p.xxv]{govtReport}  

Credit rating agencies rely on these Quantitative Analysts (known as ``quants'') to create computer models that use past market data, and current information about the market, to generate predictions of what the market might do, by using a combination of statistics, economics and computer science. These models, combined with other research, are used by the agencies to help determine the rating they give to a security or bond.\cite{quantsRole} 

In an a front-page article done by Wired Magazine, they describe the Gaussian Copula Function, which was developed as a way to model the correlation between defaults without having to use historical data, as the source of these inaccurate models.  The problems it caused wasn't the fault of the function, but rather an improper use of the function.  Paul Wilmott, a quantitative finance consultant and lecturer said that, ``correlations between financial quantities are notoriously unstable.''  According to the article, rating agencies such as Moody's used this function as a way to not worry about the underlying parts of a security when determining it's risk, but rather just use the correlation number to determine its rating. \cite{wiredFormula}

%%%%%%%%%%%%%%%%%%%%%%%%%
%%% Research Question %%%
%%%%%%%%%%%%%%%%%%%%%%%%%

\section{Research Question}

\textit{Is it ethical for Standard \& Poor's and Moody's to rate and act as gatekeepers for industries in which they stand to benefit based on the ratings they give?}
\newline

\subsection{Why is this important?}

The ratings on bonds issued by these agencies affect more than just the parties directly involved. The FCIC report showed that in late 2008, the sudden downgrade in MBS and CDO ratings by the credit rating agencies (Standard \& Poor's, Moody's and Fitch's) was the leading cause for the crisis. \cite{huffCreditCause, govtReport}  This resulted in the market dropping 41\% in eight weeks \cite{marketWatch} and sent the US into a long-term recession in which 39\% of households experienced unemployment or other economic hardship. \cite{collapseImpact}  The credit agencies are a crucial part of the world economic structure; without a rating by one of the big three agencies, it is almost impossible to sell any sort of security.  In fact, most securities are required by law to be rated by two agencies before they can be bought or sold. \cite{wpMoodies}  This means that the world financial market cannot function currently without these companies.  Because of the serious role Standard \& Poor's and Moody's played in the recent financial crisis, and their role as ``gatekeepers'' of the financial markets, answering the above stated question is extremely important not only for software engineers and financial professionals, but for the general public.    

%%%%%%%%%%%%%%%%%%%%%%%%%
%%% Extant Arguments from External Sources %%%
%%%%%%%%%%%%%%%%%%%%%%%%%

\section{Extant Arguments}

Due to the severe consequences the financial crisis has had on the world economy and it's citizens, the arguments for and against credit rating agencies have been taken all the way up to the US Supreme Court.  Up until the financial crisis, credit rating agencies were under the legal protection of the First Amendment and protected under the Securities Act of 1933.  After the collapse, these arguments began to deteriorate giving way to strong arguments against credit rating agencies.  The next two sections outline both sides of the argument.
\newline
\newline

\subsection{Arguments For}

\textit{It is ethical for private rating agencies to rate credit}

\subsubsection{First Amendment Rights}

Historically, credit rating agencies have been protected from liability for their ratings under the First Amendment to the Constitution. \cite{CivilLiability} The First Amendment states that ``Congress shall make no law \ldots abridging the freedom of speech, or of the press \ldots'' \cite{firstAmendment}  The rating agencies have relied on the First Amendment of the Constitution as their primary defense in numerous court cases against Washington regulators and angry investors. Their position is that the ratings they provide are merely opinions which are protected under free speech. \cite{CivilLiability, CRS} In a recent New York Times article, Floyd Abrams, renowned First Amendment lawyer and defendant for credit rating agencies in numerous cases, said that the ratings are just educated opinions about the quality of bonds, not guarantees that the bonds will or will not default.  Furthermore, he says that credit rating agencies should have the same rights of as journalists since they are simply providing an opinion on the rating of an investment based on researched facts.  Therefore, they cannot be held liable for damages when their rating tuned out to be wrong.  \cite{nyTimesFirstAmendment}  As such, Moody's standard disclaimer on all of it's ratings is ``the ratings \ldots are, and must be construed solely as, statements of opinion and not statements of fact or recommendations to purchase, sell, or hold any securities.'' \cite[p.120]{govtReport}
\newline
\newline

\subsection{Arguments Against}

\textit{It is not ethical for private rating agencies to rate credit}

\subsubsection{Poor Performance and Quality}

Competition in the credit rating market would ideally be based on three things: rating quality, price and service. \cite[p. 210]{govtReport} According to Andrew Kimmel, Moody's Chief Credit Officer, ``Unfortunately \ldots rating quality is proving the least powerful given the long tail in measuring performance.''  The effect of this lack of emphasis on quality is shown through the performance of Moody's ratings.  Of the thousands of securities given a rating of `AAA' in 2006, almost 90\% of them were downgraded within a year. \cite{ratingEthics, govtReport}

These downgrades are not without consequence.  According to the New York Times, Calpers, the state of California's pension fund worth over \$173 billion in assets, filed suit in July of 2009 against the three largest credit rating agencies (Standard \& Poor's, Moody's and Fitch.) The suit was for ``negligent misrepresentation'' of information in its' ratings of certain investments in which Calpers invested around \$1.3 billion in 2006.  They invested in these securities based on the rating agencies `AAA' rating, of which the suit says the ratings ``proved to be wildly inaccurate and unreasonably high'' and the method in which the ratings were determined was ``seriously flawed in conception and incompetently applied.''  this resulted in over a billion in losses. \cite{nyTimesCalpers} 

\subsubsection{Conflict of Interest}

According to a report published in 2008 by the SEC on the investigation of credit rating agencies following the credit crisis, the ``issuer pays'' business model used by the credit rating agencies has an inherent conflict of interest. \cite[p.23]{secCRAreport}  More specifically, the Securities and Exchange Act of 1934 states that a conflict of interest exists when they are ``being paid by issuers or underwriters to determine credit ratings with respect to securities or money market instruments they issue or underwrite.'' \cite[Rule 17g-5(b)(1)]{SEA} 

The SEC report cites that the conflict of interest arises in part because of the credit rating agencies incentive to generate business from the firms needing the ratings. \cite{secCRAreport}  For example, whenever Goldman Sachs purchases mortgages from a bank to package into it's securities, the resulting security tranches, which are basically levels of risk in the investment, must be rated by a credit rating agency such as Moody's.  Moody's charges a percentage fee per rating, which the FCIC reports example payments to be around \$200,000 for a typical mortgage-backed security and between half a million to \$850,000 for a complex CDO.\cite[p.146]{govtReport} It is important to note that financial firms need good ratings on investments securities - the higher the rating, the better the price.  

Two major problems come from this ``issuer pays'' model.  The first is that agencies had incentive to rate as many securities as they could - often unsolicited. \cite[p.60]{gatekeepers} The agencies also had incentive to meet the financial firms expectations of high ratings.  Agencies were only compensated for deals that were accepted by the issuer. \cite[p.210]{govtReport} For example, an analyst from one of the agencies said in an email regarding the rating of a CDO, ``I am trying to \ldots determine at this point if we will suffer any loss of business because of our decision.'' He goes onto further say that the CDO rating team ``didn't agree with you [on the rating determined] because they believed it would negatively impact business.'' \cite[p. 26]{SEA} 

The second problem arises with the financial firms incentive to get a higher rating.  According to the FCIC report and numerous news organizations, both Standard \& Poor's and Moody's received pressure from financial firms \cite[p.xxv]{govtReport}, specifically the eight banks mentioned in the facts section 1.2.  As a result of this pressure, many ratings issued by the agencies were higher than they should have been, and ultimately many of them were downgraded between 2006 - 2007 \cite{ratingEthics, huffCreditCause}  This pressure was applied by giving the agencies short time frames to research the rating and threatening loss of business to competitors. \cite[p.210]{govtReport}  Richard Michalek, former Moody's Vice President, when asked about whether they received pressure from investment banks, said ``Oh God, are you kidding? All the time. I mean, that's routine \ldots they would threaten you all of the time.'' \cite[p.210]{govtReport}

%The credit rating markets, as noted in the facts section 1.2, Standard and Moody's hold around 40\% market share each.  According to Warren Buffet, one of the worlds richest men, and 50.5\% owner of Moody's, says that this ``natural duopoly'' with the credit rating agencies, creates a very attractive business. \cite[]{govtReport}


%%%%%%%%%%%%%%%%
%%% Analysis %%%
%%%%%%%%%%%%%%%%
\section{Analysis}
In order to determine whether or not Standard \& Poor's (S\&P) and Moody's are ethical in rating credit the same way they have been in the past, this paper will use the Software Engineering (SE) Code of Ethics as well as Act-Utilitarianism, Deontology, and Rawlsian Justice.  To be able to apply the SE code of ethics, an individual or organization must be considered ``Software Professionals'' as defined by the SE code. The SE code states that it applies to:

\textit{Professional software engineers, including practitioners, educators, managers, supervisors and policy makers, as well as trainees and students of the profession.}
\newline

Although S\&P and Moody's are primarily known as financial companies, the are included in this definition since they develop software to analyze and aggregate investment data \cite{SnP, Moodys} as well as statistical and financial computer models to aid in the analysis of this data. \cite{govtReport} Not only do they provide software for investors and analysts, they use Quantitative Analysts (known as ``quants'') to develop models which are used to predict market trends, determine correlations between securities, and rate securities. \cite{quantsRole, govtReport, wiredFormula}  Because they develop, distribute, and use software and they employ software engineers, S\&P and Moody's can be evaluated according to the rules laid out in the SE code of ethics. 

\subsection{The Public Good - SE Code 1.02}
This first part of the code that is applicable is Rule 1.02, which states
 
\begin{framed}
\noindent
   \textbf{SE Code 1.02: }     
   \newline
   Moderate the interests of the \underline{software engineer}, ..., \underline{the client} and \underline{the users} with the \underline{public good.} \cite{SEcode}
\end{framed}

Rule 1.02 deals with the interests and conflicts that may exist between involved individuals and the overall public good.  When this rule is applied directly to the question asked by this paper, the phrase ``software engineer'' refers to Standard \& Poor's and Moody's since they are the investment service provider, meaning they aggregate data regarding investments using research and computer models to rate these investments. The ``client'' refers to investment banks such as Goldman Sachs or any other other customer of the rating agencies, since they pay for the ratings and use their investment services.  The ``users'' refer to any investor or individual who uses the ratings as a guide for their investments.  Substituting these specific definitions into rule 1.02 would state:

\begin{framed}
\noindent
   \textbf{Substituted SE Code 1.02: }  
   \newline
   Moderate the interests of \underline{Standard \& Poor's}, \underline{Moody's}, \underline{Goldman Sachs}, and \underline{all investors} with the \underline{public good}.
\end{framed}

In order to determine whether or not the above mentioned parties acted with the public good, we need to define what public good means.  In this case, public good is defined in the SE code as ``... to consider how the public, if reasonably well informed, would view their decisions; to analyze how the least empowered will be affected by their decisions.'' \cite{SEcode}  This definition touches on multiple issues which can be analyzed using Act-Utilitarianism and Rawlsian Justice. 

Act-Utilitarianism can be summarized as: the best possible outcome for the greatest number of people.\cite{utility}  Interpreting the substituted Rule 1.02 according to this definition would mean that the mentioned companies must consider the well-being of the uninformed public when making decisions and make sure they do not hurt the general public with their actions. 

Rawlsian Justice, specifically the Second Principle of Justice Part 1, states that ``social and economic inequalities should be arranged so that they are to the greatest benefit of the least advantaged persons'' \cite{rawlsian} Using this principle to interpret Rule 1.02, credit rating agencies must act in such a way that their actions benefit the least advantaged, or in the words of the SE Code, the ``least empowered.''  \cite{SEcode}

\subsubsection{Public Good}

To show if the greatest good for the greatest number of people was achieved, there needs to be a way to quantify and calculate the ``good.''  In the case of financial markets and the ethical question presented, the ``good'' can be considered personal wealth and economic status.  They also must perform all aspects of their business not only based solely on their interests in making money for the company and its' shareholders, they must also take into consideration the interests and well being of the uninformed public.  So in order to determine if the credit rating agencies acted towards the public good in the years leading up to the financial crisis, the companies wealth (revenues and market share) and economic status (stock prices and company valuation) should reflect that of the general public.

As cited previously, the FCIC report showed that the credit rating agencies (specifically Standard \& Poor's and Moody's) were major contributors to the cause of the crisis. \cite{govtReport, huffCreditCause}  They were shown to have caved into pressure by major investment banks to give investment grade ratings to junk bonds, while also using flawed and outdated computer models to generate the information used to rate these securities. \cite[p.~xxv]{govtReport}  This resulted in massive downgrades on the ratings of thousands of mortgage-backed securities: 83\% of Moody's `AAA' rated securities in 2007 were downgraded. \cite[p.~xxv]{govtReport} The downgrade of these ``safe investment'' securities is what ultimately tipped the scale for the financial crisis, which brought the US into a long term recession costing over 11 trillion in household wealth and millions of home foreclosures. The FCIC described them in reference to the financial crisis, as ``essential cogs in the wheel of financial destruction.'' The decisions and actions of Moody's and Standard \& Poor's directly affected the lives of millions of the ``least empowered'' and, as a result, contributed to the cause of the financial crisis. Therefore, according to both Act-Utilitarianism and Rawlsian Justice, it is not ethical for Standard \& Poor's and Moody's to rate credit since there is clear evidence that they are not moderating their own interests with the interests of the public.

\subsubsection{Profits and the SE Code}

Although they acted unethically according to Act-Utilitarianism and Rawlsian Justice, it does not fully prove that Standard \& Poor's and Moody's violated Rule 1.02 of the SE Code.  To show this, the companies must have also held their interests above the publics, meaning they were striving for increased revenues and profits regardless of the impact it had on the general public.  Because Standard \& Poor's is a private company and they are not required to release financial information such as revenue and profits, I will be focusing on Moody's and its' financials for data while determining the ethics according to the SE Code.

In 2000, Moody's Corporation went public after a long history of being a research focused institution.  Erik Kolchinsky, a former managing director at Moody's, commented on the shift in company culture after the public listing, saying ``it went from [a culture] resembling a university academic department to one which values revenues at all costs.'' \cite[p.207]{govtReport}  This new value can be seen clearly when looking at the financial data over the seven years leading up to the financial collapse.  According to the FCIC report, from 2000 to 2006, the companies revenues grew from \$602 million to over \$2 billion, with revenues from rating mortgage-backed securities and CDO's (Collatorized Debt Obligations) grew from \$199 million in 2002 to \$887 million in 2007. Their profits margins grew from 26\% to 37\% \cite[p.149]{govtReport}

\subsubsection{Conclusion: SE Code 1.02}

As a publicly traded company, Moody's had a duty to their shareholders to make them money by increasing revenues and profits. Because of this, there is nothing ethically wrong with any of the data listed above.  Where an ethical question does arise is how Moody's achieved quadrupling their revenues in only seven years.  The biggest factor was the introduction of structured finance, such as mortgage-backed securities and CDO's, which accounted for 44\% of Moody's revenue, and due to the complexity of these investments, the rating given was often the only thing that investors looked at. \cite[p.149]{govtReport}  Because the agencies were paid for each rating they gave, they had an incentive to increase the quantity or ratings, often, as shown previously, at the cost of rating quality.  Former Managing Director at Moody's, Jeremy Fons, said that ``the firm became so focused on revenues \ldots that they willingly looked the other way, trading the firm's reputation for short-term profits.'' \cite[p.207]{govtReport} This loss of quality was realized when Moody's was forced to downgrade the ratings of almost 90\% of it's previously `AAA' rated securities in 2007.  Standard \& Poor's and Fitch's also experienced a similar decline in rating quality. \cite[p.212]{govtReport} 

Because the credit rating agencies sacrificed rating integrity and quality for increased revenue, Standard \& Poor's and Moody's did not act in-line with the interests of the public good and therefore, according to SE Code 1.02, acted unethically.


\subsection{Avoid Deceptive Financial Practices - SE Code 4.04}

The previous sections briefly describe the credit rating agencies actions and how they ultimately led the 2008 financial crisis.  The next rule that applies is 4.04, which relates to matters of judgment.  It states:

\begin{framed}
\noindent
   \textbf{SE Code 4.04: } 
   \newline
   [Do] Not engage in \underline{deceptive financial practices} such as \underline{bribery}, double billing, or other \underline{improper financial practices.} \cite{SEcode}
\end{framed}

This rule describes how software professionals should act in financial matters.  Given that credit rating agencies are primarily financial companies who use and develop computer models to determine the ratings of investments \cite[p.11]{govtReport}, ethical behavior in this area is especially important.  

4.04 describes deceptive financial practice as ``bribery, double billing, or other improper financial practices.''  Of these, double billing is not applicable to this topic, but bribery and other improper financial practices can be seen in Moody's and Standard \& Poor's actions.  According to the FCIC, SEC and the Congressional Research Service, although the agencies revenues and profits increased drastically, they were short staffed and ``undermanned to effectively accommodate the overwhelming volume of structured finance business.'' \cite[p. 7]{CRS}\cite{govtReport}\cite{secCRAreport}  If the credit rating agencies were private companies that had little impact on the general public or businesses around them, there would be nothing deceptive or improper about poorly allocating company resources - it would just be a poorly run business.  But because the under-staffing had a negative impact on rating quality, as will be shown below in the following analysis, and because the agencies actions have a big impact on the US and World financial markets, as shown in the analysis above, ``Improper financial practices'' will be defined as poor allocation of company resources.  Substituting these specific terms:

\begin{framed}
\noindent
   \textbf{Substituted SE Code 4.04: } 
   \newline
   Moody's and Standard \& Poor's should not participate in \underline{deceptive financial practices} such as \underline{bribery} or \underline{poor allocation of company resources.}
\end{framed}

On the surface, it seems as though Moody's and Standard \& Poor's did not participate in any sort of improper financial practice.  But when we look closer at the actions of these companies in the years leading up to the financial crisis, two areas of interest arise in which they participated in the improper financial practices.  

\subsubsection{Pressure from Internal Management and Investment Banks}

The first improper practice was bribery.  Bribery is the giving of a bribe - which is defined by Merriam Webster as ``money or favor given or promised ... to influence ... judgment'' or ``something that serves to induce or influence.'' \cite{bribeDef}  Although it seems that no exchange of money or favors was given or received by Standard \& Poor or Moody, when we apply the second definition of bribe to agencies actions, it becomes clear that the banks did ``serve to induce or influence'' when they applied pressure to alter ratings of securities.  They then accepted the ``bribe''\cite{bribeDef} by altering their ratings based on this pressure.  In order to be considered a bribe though, the agencies must have gained something from this.

As shown in the Extant Arguments, section 3.2.2, credit rating agencies needed to have rated deals, meaning the investment banks accepted the rating given, in order to receive their fee. \cite[p.210]{govtReport}  Investment banks, specifically the eight largest including Goldman Sachs, Citi, JPMorgan Chase, etc. \cite{ratingEthics}, pressured the credit rating agencies using unreasonably short time frames and threats of losing business, and the rating agencies were complaisant about it.  According to a managing director at Moody's, Moody's ``knew that they were being bullied into caving in to bank pressure \ldots Moody's allow[ed] itself to be bullied.''   ``They willingly played the game.'' \cite[p.210]{govtReport} The investment banks were ``serving to induce or influence''\cite{bribeDef}, the ratings of the securities, and the credit rating agencies, rather than rejecting the ``bribe'', they ``willingly played the game.'' 
%% NEED TO FINISH

\subsubsection{Understaffed and Overworked}

The second improper practice was the poor allocation of company resources.  Although Moody's revenues quadrupled between 2000 and 2007, the company and its employees were understaffed and under-resourced, or in the words of the FCIC investigators, ``lack of resources to do the job despite record profits.'' \cite[p.xxv]{govtReport}  According to the SEC's investigation of the three credit rating agencies in 2008, all areas of the agencies did not suffer from insufficient resources, just the MBS (mortgage backed securities) and CDO rating departments. \cite{secCRAreport}  In the report, they share one analysts emails saying ``our staffing issues, of course, make it difficult to deliver the value that justifies our fees'' and, ``Just too much work, not enough people, pressure from company, quite a bit of turnover and no coordination of the non-deal 'stuff' they want us and our staff to do.'' \cite{secCRAreport}  Unlike most other markets, where under-performing companies lose market share and suffer, Moody's profits and revenue just kept increasing as shown in section 1.2 of this paper.  But this staffing issue did more than just cause stress for employees and provide less value to the customer, it caused the inevitable slip in rating quality without the customer (investment banks) caring about the quality.   

According to one of the CDO groups senior managers at Moody's, the group was ``under-resourced'' and had trouble ``retaining and recruiting good staff,''  ``We had almost no ability to do meaningful research.'' \cite[p.149]{govtReport}  Despite the companies record profits (specifically the CDO division, which accounted for almost 44\% of the companies revenue) - the short-staffed, inexperienced CDO group felt intense pressure from banks and internal management to give high ratings in order to keep market share and revenues high. \cite[p.149-150]{govtReport}

\subsubsection{Conclusion: SE Code 4.04}

Principle 4, specifically rule 4.04, is included in the SE code to address the ethics behind company management in order to facilitate the ethical development of the companies products.  Through the above analysis, it has been shown that the credit rating agencies poor internal management and allocation of company resources facilitated an environment where revenues and market share mattered more than the quality of the ratings.  Also, outside pressure from investment banks, and the subsequent action of caving to this pressure, shows that the banks ``served to induce or influence'' or ``bribe'' the rating agencies, making their actions, according to SE code 4.04 unethical.

 
\subsection{Quantatative Models - Principle 3}

\begin{framed}
\noindent
\textbf{SE Principle 3:  }
\newline
Software engineers shall ensure that their products and related modifications meet the highest professional standards possible.
\end{framed}

The introduction of new structured finance products, specifically mortgage-backed securities (MBS) and collatorized debt obligations (CDO), created by investment banks such as Goldman Sachs and JP Morgan Chase, made determining the quality of these new investments incredibly complex. \cite[p.43]{govtReport} This created an opportunity for the rating agencies to expand their market by rating these CDO's and MBS's with their easy to understand rating scale. \cite[p.43]{govtReport}  While it created a new market, it also created the need for a new way to determine the rating of these investments since traditional methods of research were not enough to accurately rate these securities.\cite[p.43]{govtReport}  So they used Quantitative Analysts to create computer models to determine the risk, and ultimately, the rating of the investment. \cite{wiredFormula}\cite[p.120]{govtReport}  

These credit rating agencies, specifically Moody's, can be held to Principle 3 of the SE code of ethics since they created the computer models used to rate securities.  Therefore, these models can be considered the ``product''\cite{SEcode} mentioned in Principle 3 because it is essential to their main business.  Specifically, rule 3.14 of the SE code of ethics will be used in the ethical analysis of Moody's rating models.  It states that software engineers should:

\begin{framed}
\noindent
   \textbf{SE Code 3.14: } 
   \newline
   Maintain the \underline{integrity of data}, being sensitive to \underline{outdated} or \underline{flawed occurrences}.
\end{framed}

Rule 3.14 describes the considerations that software engineers need to make when dealing with data used in and by software.  Breaking up the rule into smaller parts, the ``integrity of data'' \cite{SEcode} is defined as ``the quality of correctness, completeness, soundness and compliance with the intention of the creators of the data.'' \cite{dataIntegrityDef} Thus, ``integrity of data'' in Moody's case, can be substituted to say ``integrity of Moody's financial data,'' while ``outdated or flawed occurrences'' \cite{SEcode} can be substituted to say ``outdated and flawed financial data.''  Therefore, after substituting these values into SE code 3.14, it states:

\begin{framed}
\noindent
   \textbf{Substituted SE Code 3.14: } 
   \newline
   Maintain the \underline{integrity of Moody's financial data}, being sensitive to \underline{outdated and flawed financial data}.
\end{framed}

In order to show whether or not Moody's maintained the integrity of its financial data, we have to examine the outcome and performance of the ratings generated by these models and the effect the data given to these models had on these ratings performance.  Because ``integrity of data''\cite{SEcode} can be seen as intrinsically good, meaning every rational being would desire `correct, complete and sound data \cite{dataIntegrityDef}, the actions of Moody's can be analyzed ethically using Deontology as well as rule 3.14.  \cite{kant}

First though, before performance can be evaluated, I will describe how and with what Moody's rates mortgage-backed securities (MBS) and collatorized debt obligations (CDO).

\subsubsection{Mortgage-backed Securities}

Historically, Moody's has rated mortgage-backed securities using three different models: the standard residential mortgage-backed security model, M3 Prime, and M3 Subprime. \cite[p.120]{govtReport}  All three of these models use market trends, legal and regulatory factors, security-specific factors and firm specific factors to help predict the level of safety the rated mortgage-backed security has.  Also, they use loan-to-loan ratios, borrower credit scores, mortgage originator quality, loan terms, and other general loan information rather than analyzing the specifics of each individual loan within a given security. \cite[p.120]{govtReport}  The M3 Prime and M3 Subprime models differ from the standard MBS rating model in that they are specialized towards rating certain types of securities.  M3 Prime allows for more automation of the rating process since the securities it rates are packaged with high quality loans, while M3 Subprime was created to handle securities packaged with subprime mortgages, so it was designed to be more conservative in its rating output. \cite[p.120-121]{govtReport}  

Moody's would then take the given data about the MBS and its' packaged loans and plug it into the model which would run it against 1,250 different scenarios to emulate different possible events.  From there, analysts would take the rating generated by the model and analyze it against external research to determine the final credit rating. \cite[p.121]{govtReport}

\subsubsection{The Gaussian Copula Function and Collatorized Debt Obligations}

On August 10 2004, Moody's added to their CDO rating model a function called the ``Gaussian copula function.'' \cite{ftGaussianCopula}  According to a March, 2009 cover story by Wired Magazine, this function, created by David X. Li, was developed as a way to model default correlation without any historical default data. \cite{wiredFormula}  The correlation number produced by this function essentially ignored the correlation between each and every individual loan, and instead determined the correlation using other financial measures, specifically the credit default swap markets. \cite{wiredFormula}  This immediately simplified the rating of CDOs by eliminating the need to diversify with other assets such as student loans or credit card debt, which was what Moody's did prior to August 2004 and was considered good practice by all three credit rating agencies. \cite{ftGaussianCopula}  According to articles in both Wired and the Financial Times, rating agencies instead just used their new models to generate a correlation number which was then used to package anything into securities - including the deadly subprime mortgage. \cite{ftGaussianCopula, wiredFormula}  

In order to determine the rating of the CDO, Moody's used its' own ratings of the mortgage-backed security where the loan originated from. ``We took the rating that had already been assigned by the [mortgage-backed securities] group,'' testified Gary Witt, one of Moody's team managers for the CDO group, to FCIC investigators. \cite[p.146-147]{govtReport}  Using the correlation number generated from the copula function, and piggy-backing on the ratings previously assigned to the MBS, they were able to rate thousands of these CDOs `AAA.' The idea behind this was, if the loans within the CDO have a low default correlation, the CDO is diversified since the failure of one mortgage would not cause the failure of another.  As a result of this, the CDO market boomed, growing from \$275 billion in 2000 to \$4.7 in 2006. \cite{wiredFormula}
 
\subsubsection{The Result}

As stated in the Facts section 1.2, of the thousands of `AAA' ratings given out between 2005 and 2006, 83\% were downgraded within a year, and by the end of 2008, more than 90\% of all CDOs were downgraded\cite[p.224]{govtReport} - ultimately tipping the scale for the financial collapse. \cite[p.~xxv]{govtReport}  Although the poor quality of ratings were shown previously in section 3.2.1, it was also caused by these models.  The FCIC report states that the computer models were ``divorced from reality - and [depended] on ever rising housing prices.'' When the housing market dropped 38\% nationally in 2007, the models couldn't handle the drop.  In all 1,250 scenarios run by the model, housing prices were shown as increasing by 4\% per year. \cite[p.120-121]{govtReport}

The FCIC report also notes that the models were ignoring the ``deteriorating quality'' of loans being packaged into the securities it was rating.  With the increases in subprime mortgage lending and, consequentially, increases in creation of subprime-backed securities and the resulting CDOs created from these securities, no one really knew what was in these CDOs and securities.  Ultimately, when the housing bubble burst, most of these subprime, `AAA' rated securities were forced to be downgraded. \cite[p.121]{govtReport}

\subsubsection{Conclusion: SE Code 3.14}

How is this violating the integrity of data?  According to Gregg Berman, co-founder of a risk-management group at JPMorgan, ``There was a willful designing of the systems to measure the risks in a certain way that would not necessarily pick up all the right risks.'' \cite{nyTimesQuants}  This is shown through the models assumed ever-increasing housing prices and ignored deteriorating quality of the ratings.  Extant Arguments, section 3.2.2, shares some detailed facts about what the agencies had to gain from having subprime-backed securities.  Because ``integrity of data'' \cite{SEcode} is defined as ``the quality of correctness, completeness, soundness and compliance with the intention of the creators of the data,'' \cite{dataIntegrityDef} and Moody's did not take into consideration decreases in housing prices and decreasing quality of securities, they did not meet a level of quality with completeness or correctness to meet the definition of ``integrity of data.'' Therefore they acted unethically according to the standards of Deontology and the SE Code 3.14 since they did not ``maintain the integrity of data.'' \cite{SEcode}


%%%%%%%%%%%%%%%%
%%% Conclusion %%%
%%%%%%%%%%%%%%%%
\section{Conclusion}

Below is a table that summarizes the ethical analysis from Section 4.
\newline

\begin{tabular}{|l|l|}
\hline
\multicolumn{2}{|c|}{Summary of SE Code Analysis} \\
\hline
\textbf{SE Code Rule} & \textbf{Result} \\ 
\hline
SE Code 1.02 & Unethical \\ 
\hline
SE Code 4.04 & Unethical \\
\hline
SE Code 3.14 & Unethical \\
\hline
\textbf{Final Conclusion} & \textbf{Unethical} \\
\hline
\end{tabular}
\newline

As shown in the above table, Standard \& Poor's and Moody's acted unethically the rating of bonds and securities due to their disregard of the public good, inescapable conflict of interest with investment banks and other issuers, and the poor quality of their rating methods and ratings.  Because of the serious impact the actions of these companies have had on the general, uninformed public, Standard \& Poor's and Moody's must be held to a higher standard.  

Unfortunately, there isn't much anyone can do about this current situation without serious reform and rewriting of numerous laws and financial regulations.  Credit rating agencies are ingrained in the US, and world, financial system and it looks like that won't change anytime soon.  What we can hope for is increased oversight and regulation from the government and for companies, like Moody's, Standard \& Poor's and all the other companies who contributed to the financial crisis, to learn from their mistakes and not repeat history - again.

%end the two column format
\end{multicols}
\newpage


%cite all the references from the bibtex you haven't explicitly cited
\nocite{*}

\bibliographystyle{IEEEannot}

\bibliography{texreport}

\end{document}
